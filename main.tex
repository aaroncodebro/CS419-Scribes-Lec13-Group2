%
% This is the LaTeX template file for lecture notes for CS294-8,
% Computational Biology for Computer Scientists.  When preparing 
% LaTeX notes for this class, please use this template.
%
% To familiarize yourself with this template, the body contains
% some examples of its use.  Look them over.  Then you can
% run LaTeX on this file.  After you have LaTeXed this file then
% you can look over the result either by printing it out with
% dvips or using xdvi.
%
% This template is based on the template for Prof. Sinclair's CS 270.

\documentclass[11pt, twosides]{article}
\usepackage[utf8]{inputenc}
\usepackage{graphicx}
\usepackage{graphics}
\usepackage{amsmath}
\usepackage{amsfonts}
\usepackage{amssymb}
\usepackage{amsthm}
\usepackage{xcolor}
\setlength{\oddsidemargin}{0.25 in}
\setlength{\evensidemargin}{-0.25 in}
\setlength{\topmargin}{-0.6 in}
\setlength{\textwidth}{6.5 in}
\setlength{\textheight}{8.5 in}
\setlength{\headsep}{0.75 in}
\setlength{\parindent}{0 in}
\setlength{\parskip}{0.1 in}

%
% The following commands set up the lecnum (lecture number)
% counter and make various numbering schemes work relative
% to the lecture number.
%
\newcounter{lecnum}
\renewcommand{\thepage}{\thelecnum-\arabic{page}}
\renewcommand{\thesection}{\thelecnum.\arabic{section}}
\renewcommand{\theequation}{\thelecnum.\arabic{equation}}
\renewcommand{\thefigure}{\thelecnum.\arabic{figure}}
\renewcommand{\thetable}{\thelecnum.\arabic{table}}

%
% The following macro is used to generate the header.
%
\newcommand{\lecture}[4]{
%   \pagestyle{myheadings}
   \thispagestyle{plain}
   \newpage
   \setcounter{lecnum}{#1}
   \setcounter{page}{1}
   \noindent
   \begin{center}
   \framebox{
      \vbox{\vspace{2mm}
    \hbox to 6.28in { {\bf CS 419M Introduction to Machine Learning
                        \hfill Spring 2021-22} }
       \vspace{4mm}
       \hbox to 6.28in { {\Large \hfill Lecture #1: #2  \hfill} }
       \vspace{2mm}
       \hbox to 6.28in { {\it Lecturer: #3 \hfill Scribe: #4} }
      \vspace{2mm}}
   }
   \end{center}
   \markboth{Lecture #1: #2}{Lecture #1: #2}
}

%
% Convention for citations is authors' initials followed by the year.
% For example, to cite a paper by Leighton and Maggs you would type
% \cite{LM89}, and to cite a paper by Strassen you would type \cite{S69}.
% (To avoid bibliography problems, for now we redefine the \cite command.)
% Also commands that create a suitable format for the reference list.
% \renewcommand{\cite}[1]{[#1]}
% \def\beginrefs{\begin{list}%
%         {[\arabic{equation}]}{\usecounter{equation}
%          \setlength{\leftmargin}{2.0truecm}\setlength{\labelsep}{0.4truecm}%
%          \setlength{\labelwidth}{1.6truecm}}}
% \def\endrefs{\end{list}}
% \def\bibentry#1{\item[\hbox{[#1]}]}

%Use this command for a figure; it puts a figure in wherever you want it.
%usage: \fig{NUMBER}{SPACE-IN-INCHES}{CAPTION}
% \newcommand{\fig}[3]{
% 			\vspace{#2}
% 			\begin{center}
% 			Figure \thelecnum.#1:~#3
% 			\end{center}
% 	}
% Use these for theorems, lemmas, proofs, etc.
\newtheorem{theorem}{Theorem}[lecnum]
\newtheorem{lemma}[theorem]{Lemma}
\newtheorem{proposition}[theorem]{Proposition}
\newtheorem{claim}[theorem]{Claim}
\newtheorem{corollary}[theorem]{Corollary}
\newtheorem{definition}[theorem]{Definition}
% \newenvironment{proof}{{\bf Proof:}}{\hfill\rule{2mm}{2mm}}

% **** IF YOU WANT TO DEFINE ADDITIONAL MACROS FOR YOURSELF, PUT THEM HERE:

\begin{document}
%FILL IN THE RIGHT INFO.
%\lecture{**LECTURE-NUMBER**}{**DATE**}{**LECTURER**}{**SCRIBE**}
\lecture{13}{Introduction to Non Linearity and Gradient Descent}{Abir De}{Group 13}
%\lecture{x}{Title}{Abir De}{Group y}

\section{Introduction To Non-Linearity}
Linearity here refers to modelling the Output as a linear function of the corresponding input, $ \bf{y_{i}} \propto \bf{x_{i}}$. \newline
We turn our attention to non-linear models as linear models aren't as accurate as we would like them to be, or rather do not \emph{model} the output accurately. Here accuracy is used rather loosely, it just refers to the model not being able to trace the required output.

\section{What are the choices for non linear models?}
\subsection{Polynomial Approximations}
Here we try to model the output as a polynomial function 
\[y= \sum_{i=1}^{\infty} a_{i}x^{i} \] \
The space spanned by 1, x, $x^{2}$... does not include the set of all function and hence we can say some linear combination of these will not perfectly fit certain outputs. We also see that while modelling log(x), the polynomial fit causes issues at x=0. Moreover, as the exponents of x grow bigger the terms shoot up and cause stability issues.

\subsection{Universal Approximators}
\begin{theorem}
\textbf{Universal approximation theorem (non affine activation, arbitrary depth and constrained width)}: Let $\chi$ be a compact subset of $R^{d}$. Let $\sigma$: R $\xrightarrow{}$ R be any non-affine continuous  which is continuously differentiable at at least one point, with nonzero derivative at that point. Let $N_{d,D:d+D+2}$ denote the space of feed-forward neural networks with d input neurons, D output neurons and an arbitrary number of hidden layers with d+D+2 neurons such that every input neuron has activation function $\sigma$ and every output neuron has identity as activation function as identity, with input layer $\phi$ and output layer $\rho$. Then given an $\epsilon>0$, and any function belonging to the set of continuous functions from X $\xrightarrow{}R^d$ there exists an approximator $\hat{f}$ belonging to $ N_{d,D:d+D+2}$ such that \[
\sup_{x \in \chi} |f(x)-\hat{f(x)}|<\epsilon
.\]
Source: Wikipedia-Universal approximation theorem

\end{theorem}
The above theorem states that there exists an approximator/neural network under appropriate conditions which can approximate any continuous function. However, it is important to note that this theorem does not give us a method to obtain such an approximator for a given function. One such approximator is an LRL (Linear-ReLU-Linear) unit.  
Here, the middle ReLU layer is sandwiched between 2 Linear layers.
That is, $z_1= w.X_{input}+b$ \newline
$z_2= ReLU(z_1)$ \newline
$z_3= p.z_2+q$ where w and p are weight vectors of appropriate dimension (depending on input) and b,q are respective bias terms.
Since, this satisfies the conditions of universal approximation theorem we are guaranteed to have an approximator of this form, but the number of layers (LRL units) so that we get a reasonably close approximation are usually found out by trial and error.

%\begin{itemize}
%    \item A set of images $\xrightarrow{m_\theta}$ identify the objects
%    \item A set of paragraphs $\xrightarrow{m_\theta}$ identify the topics 
%\end{itemize}
%where $m_\theta$ denotes a model with parameters $\theta$. The objective is to devise a model which learns to do a task. The following question can be posed:
%\begin{flushleft}
%Q. \textbf{What is the underlying thesis/concept behind the hope that it will be %possible to train the model $m_\theta$ to get an accurate output?}\\

%Ans. \color{blue}
%\textbf{The law of large numbers}
%\end{flushleft}

\section{Introduction to Gradient Descent}
Given an arbitrary function $f(x)$, parametrized by $\theta$,and a data set $\mathcal{D}$ we wish to find the optimal parameter $\theta^{*}$ such that $y \approx f_{\theta^{*}}(x)$. More specifically, we wish to minimize the following objective 



\begin{equation}
\min_{\theta} l_{\theta} = \min_{\theta} \sum_{i \in \mathcal{D}}^{} (y_i - f_{\theta}(x_i))^2
\end{equation}

We start with an initial arbitary initial point $\theta_{0}$, which we hope to be near the local minima. Then, $\theta_{t}$ is calculated iteratively according to the following update rule, 

\begin{equation}
    \theta_{t+1} = \theta_{t} - \gamma \left(\frac{\partial  l_{\theta}}{\partial \theta} \right)_{\theta = \theta_{t}}
\end{equation}

Here $\gamma$ is called the learning rate. 

Observe that, if we make $\gamma$ high, although it would increase the learning speed, but it may so happen that the point escapes local minima or fail to converge, due to oscillations around the minima. Hence, we need to find appropriate bounds on $\gamma$. 

By second order taylor approximation we have the following, 

\begin{equation}
    l_{\theta_{t+1}} = l_{\theta_{t}} + \frac{\partial  l_{\theta}}{\partial \theta}(\theta_{t+1} - \theta_{t}) + \frac{1}{2}(\theta_{t+1} - \theta_{t})^{T}H(\theta)(\theta_{t+1} - \theta_{t})
\end{equation}

\begin{equation}    
   \implies  l_{\theta_{t+1}} \leq l_{\theta_{t}} + \frac{\partial  l_{\theta}}{\partial \theta}(\theta_{t+1} - \theta_{t}) + \frac{\lambda_{max}}{2}||\theta_{t+1} - \theta_{t}||^{2}
\end{equation}

Since, $H(\theta) \leq \lambda_{max} I$ ,where $\lambda_{max}$ is the maximum of the set of eigenvalues and $I$ is the identity matrix. 

\begin{equation}
    \implies  l_{\theta_{t+1}} - l_{\theta_{t}} \leq -\gamma \left|\frac{\partial  l_{\theta}}{\partial \theta}\right|^2 + \frac{\lambda_{max}}{2}\gamma^2\left|\frac{\partial  l_{\theta}}{\partial \theta}\right|^2 < 0
\end{equation}

Since, we want the loss to decrease, ie. $l_{\theta_{t+1}} - l_{\theta_{t}} <0 $, hence we can bound the upper bound of the above inequality by 0 to get condition on $\gamma$

\begin{equation}
    \implies \gamma < \frac{2}{\lambda_{max}}
\end{equation}

Note that, if any of the eigenvalue of the hessian is very large, learning becomes very difficult due to a very low learning rate constraint. 

In case of convexity, we can also have the following inequality, 

\begin{equation}
    l_{\theta_{t+1}} - l_{\theta_{t}} \geq \left(\frac{\partial  l_{\theta}}{\partial \theta} \right)_{\theta = \theta_{t}}^{T}(\theta - \theta_{t})
\end{equation}

\begin{equation}
    \implies l_{\theta_{t+1}} \geq  l_{\theta_{t}} -\gamma\left|\frac{\partial  l_{\theta}}{\partial \theta}\right|^2
\end{equation}


We can also represent gradient descent by the following optimization problem, 

\begin{equation}
    \theta_{t+1} = \arg \min_{\theta} \frac{1}{2}||\theta - \theta_{t}||^2 + \gamma\left( l_{\theta_{t}} + (\theta - \theta_{t})^{T}\left(\frac{\partial  l_{\theta}}{\partial \theta} \right)_{\theta = \theta_{t}}\right)
\end{equation}

The first term, wants $\theta$ to be close to $\theta_{t}$, whereas the second term forces a movement in the direction of gradient. The parameter $\gamma$ controls the tradeoff between the two terms. 


\subsection{Taking data in batches}

Consider 
\begin{equation}
    l_{\theta_t} = \sum_{x \in D} (y_i - f_{\theta_t}(x_i))^2
\end{equation}

Here we have taken complete dataset while calculating loss. But the problem with this loss function is that, if we are stuck in some local minima, then we can not get out of it.

Now consider
\begin{equation}
    l_{\theta_t} = \sum_{x \in B_t} (y_i - f_{\theta_t}(x_i))^2
\end{equation}

where $B_t$ is a batch of dataset. 
Now notice that, loss function will change in every batch and hence if we are stuck at local minima, then we can get out of it.

We may also use 
\begin{equation}
    l_{\theta_t} = \frac{\sum_{x \in B_t} (y_i - f_{\theta_t}(x_i))^2}{|B_t|}
\end{equation}
where $|B_t|$ is batch size. 

But then 
\begin{equation}
    \theta_{t+1} = \theta_t - \frac{\gamma}{|B_t|} \cdot \frac{\partial l_{\theta_t}}{\partial \theta}
\end{equation}

Initially $\gamma$ was small (in orders of $10^{-3}$). $|B_t|$ is usually in orders of 100. Thus $\frac{\gamma}{|B_t|}$ will be in orders of $10^{-5}$ which is very small. We can increase $\gamma$ itself such that $\frac{\gamma}{|B_t|}$ is in appropriate range.

\begin{flushleft}
\textbf{Q. Why can't we take $\gamma = || \theta_{t-1} - \theta_{t-2}||_{l_2}?$}\\
Ans. \color{blue} Consider the case where we have divided our dataset into batches and we are training our neural network by giving batch by batch data.
\begin{equation}
    \theta_{t+1} = \theta_t - \gamma \cdot \frac{\partial l_{\theta_t}}{\partial \theta}
\end{equation}

Notice that $\frac{\partial l_{\theta_t}}{\partial \theta}$ will not be always zero as data changes according to batch. \\
Now if $\gamma$ is a non zero constant, then learning will occur whenever $\frac{\partial l_{\theta_t}}{\partial \theta}$ is non zero.\\
But if $\gamma$ is $|| \theta_{t-1} - \theta_{t-2}||_{l_2}$, learning will stop when $\theta_{t-1} = \theta_{t-2}$. Because of zero learning, $\theta$ will not be updated and we will be stuck at local minima.

\end{flushleft}


% \subsection{The Law of Large Numbers and Central Limit Theorem}
% Suppose that we have $N$ samples $X_i \sim f(\cdot), \:\:\: i = 1, \hdots, N$ drawn from an unknown distribution. Most of the times, the goal of machine learning is to estimate the distribution $f$ or its properties. For example, in the task of generating images (GAN, VAE etc), the objective of the machine learning model is to identify the underlying distribution of the data samples. Being able to learn the underlying distribution is important even for tasks like housing price prediction, image classification etc because if a test sample is sampled from a different distribution is given to the model, the model will likely fail and give incorrect results. 
% \begin{flushleft}
% Q. \textbf{What is the least that can be found about $f$ using $X_1, \hdots, X_N$?}\\

% Ans. \color{blue} Central Limit Theorem states that as long as $N$ is large, the $$\mathbb{E}_{X \sim f(\cdot)}[X] = \frac{\sum X_i}{N}$$
% The Law of Large Numbers further states that
% $$\mathbb{E}_{X \sim f(\cdot)}[X] \xrightarrow{}\frac{\sum X_i}{N}$$
% as $N \xrightarrow{} \infty$. Moreover, the random variable $Z_N = \frac{\sum X_i}{N}$ follows the normal distribution $\mathcal{N}(\mathbb{E}[X], \sigma^2)$
% with $\sigma^2 \propto \frac{1}{N}$ with
% $$\lim \limits_{N \xrightarrow{}\infty}\mathbb{E}[Z_N] = \mathbb{E}[X]$$
% \end{flushleft}
% \begin{flushleft}
% Q. \textbf{What else do we need to completely characterize $f$?}\\

% Ans. \color{blue} \textbf{We need the higher order moments!} Using these the moment generating function can be obtained.
% \end{flushleft}

% \definition{The moment generating function (MGF) $F(s)$ for a random variable $X$ is defined as the Laplace Transform of the probability density function (PDF) $f(x)$. The inverse Laplace Transform of MGF will give the PDF.
% $$F(s) = \int \limits_{-\infty}^\infty e^{-sx}\, f(x)\, dx$$
% $$f(x) = \int \limits_{-\infty}^\infty e^{sx}\, F(s)\, ds$$}

% \proposition{Using only the moments, the MGF can be determined.}
% \begin{proof}
% \begin{eqnarray*}
% F(s) &=& \int \limits_{-\infty}^\infty e^{-sx}\, f(x)\, dx\\
% &=& \int \limits_{-\infty}^\infty \sum \limits_{k = 0}^{\infty} \frac{(-s)^k\, x^k}{k!}\, f(x)\, dx\\
% &=& \sum \limits_{k = 0}^{\infty} \frac{(-s)^k}{k!}\, \mathbb{E}[X^k]
% \end{eqnarray*}
% \begin{flushleft}
% Note: The discussion about the region of convergence of the Laplace transform is out of the scope of this course.
% \end{flushleft}
% \end{proof}

% \proposition{
% Using just the $N$ samples $X_i \sim f(\cdot), \:\:\: i = 1, \hdots, N$, the PDF $f$ can be estimated using the Law of Large Numbers.
% }
% \begin{proof}
% This can be proved as follows:
% \begin{enumerate}
%     \item All moments $\mathbb{E}[X^k]$ can be estimated by invoking the Law of Large Numbers
%     \item The MGF can be found by invoking the Claim 1.2 above.
%     \item $f$ can be estimated by taking the inverse Laplace Transform of the MGF
% \end{enumerate}
% \end{proof}

% \subsection{Practice Problems}
% \normalfont
% \begin{enumerate}
%     \item  A fair coin is tossed 5 times. Find  $\mathbb{P}(\#H > \# T)$\\
%     \color{blue}It is easy to see that as 5 is odd, there is not possibility of $\#H = \#T$. Hence $\mathbb{P}(\#H > \# T) = 1/2$
%     \color{black}
%     \item $X \sim f(\cdot), Y \sim g(\cdot)$. Then $Z = X+Y \sim \:?$\\
%     \color{blue} \begin{eqnarray*}
%     \mathbb{P}(Z \leq z) &=& \mathbb{P}(X+Y \leq z)\\
%                         &=& \int \limits_{-\infty}^\infty \int \limits_{-\infty}^{z-x}f(x) \, g(y) \, dx\,dy\\
%                         &=& \int \limits_{-\infty}^\infty f(x)\, G(z-x) \, dx
%     \end{eqnarray*}
%     On differentiating, we get
%     $$f_Z(z) = \int \limits_{-\infty}^\infty f(x) g(z-x)\,dx = (f * g)(z)$$
% \end{enumerate}
% \section{Linear Algebra}
% Solve the following problems:
% \begin{enumerate}
%     \item For two square matrices $A, B$of size $n\times n$ if $AB = BA$ for all B, then show that $A = cI_n$ for some $c \in \mathbb{R}$\\
%     \color{blue}
%     Pick $B$ to be a diagonal matrix with pair-wise distinct elements. Then it can be shown that $A$ is also a diagonal matrix. Now pick $B$ to be a matrix with all ones, i.e. $B = [1]_{ij}$. As $A$ is diagonal, $AB=BA$ implies all diagonal entries of $A$ are equal i.e., $A = cI_n$ for some $c \in \mathbb{R}$
%     \color{black}
%     \item If $x^\top A x = 0\:\: \forall x \in \mathbb{R}^n$ then show that $A$ is skew-symmetric.\\
%     \color{blue}
%     On differentiating the above equation we get
%     $$(A + A^\top)x = 0 \:\: \forall x \in \mathbb{R}^n$$
%     This implies $A = -A^\top$
%     \color{black}
%     \item Show that $\text{rank}(AB) \leq \text{rank}(A)$\\
%     \color{blue} Each column of $AB$ can be viewed as a linear combination of columns of $A$. Hence, if the dimension of column space of $A$ is $r$, the dimension of column space of $AB$ cannot be more than $r$. In other words,
%     $\text{rank}(AB) \leq \text{rank}(A)$
%     \color{black}
%     \item Suppose you have a uniform sampler which samples uniformly from $[0, 1]$. Propose an algorithm which uses this uniform sampler to generate samples from any given distribution.\\
%     \color{blue}
%     Suppose we have a uniform random variable $U \sim \text{Uniform}([0, 1])$. We need to find a function $g$ such that the PDF of the random variable $g(U)$ will be same as that of the given distribution, say $f$. That is $g(U) \sim f$. Which is same as
%     \begin{eqnarray*}
%     \mathbb{P}(g(U) \leq x) &=& \mathbb{P}(U \leq g^{-1}(x))\\
%     \implies \int \limits_{-\infty}^x f(x) \, dx &=& \int \limits_{-\infty}^{g^{-1}(x)} 1 \, du\\
%     \implies F(x) &=& g^{-1}(x)\\
%     \implies g &=& F^{-1}
%     \end{eqnarray*}
%     \color{black}
% \end{enumerate}


\section{Group Details and Individual Contribution}
% Fill this part
\begin{enumerate}
    \item Aaron Jerry Ninan 190100001 - Introduction to Gradient Descent
    \item Bhavya Singhal 200010014 - Why we cant take $\gamma = || \theta_{t-1} - \theta_{t-2}||_{l_2}?$
    \item Kaiwalya Joshi 20D070043 - Taking data in batches
    \item Rhishabh Suneeth 200260040 - Universal Approximators
    \item Rishabh Ravi 200260041 - Introduction to non linear models
\end{enumerate}
\end{document}





